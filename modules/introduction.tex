%!TEX root = ../paper.tex

% ---------------------------------------------------------------------------- %
% Outline

\iftoggle{outline}{

    \begin{itemize}
        \item VMs are good
        \item Hard to build, even with VM gramworks like Truffle \& pypy
        \item Makes it hard for a new langauge to get a VM
        \item We are exploring making a VM for Grace by adapting VM for NS
        \item initial results are positive
    \end{itemize}

}

\iftoggle{oldcomments}{
	\sm{I feel this characterization is not quite accurate. At least for RPython, each interpreter gets its own VM generated. With Truffle, we indeed reuse the JVM as a host VM.}
	\sm{The use of macros seems highly over-engineered...}
	\sm{and, you might like the xspace package, I think, to avoid always having to put braces at the end}
	\sm{really not a fan of macros. This makes editing really hard!}
}

% 
% ---------------------------------------------------------------------------- %



% ---------------------------------------------------------------------------- %
% 

With Truffle and RPython, language implementation frameworks have emerged that enable us to
implement languages with the performance of state-of-the-art \vm/s by building
simple interpreters.
% \vm/ frameworks have emerged and are quickly becoming a core component to language implementation. 
% Truffle and RPython are \VM/ frameworks: 
With Truffle, \AST/ interpreters are built on top of a host \VM/, which provides for instance garbage collection and just-in-time compilation.
% which a developer uses to author a simple \AST/ interpreter and through doing so automatically gain features such, and optimizations to the \AST/; and
With the RPython toolchain, a tailored \vm/ is generated based on a simple interpreter. The toolchain adds for instance garbage collection and a tracing
just-in-time compiler to enable competitive performance.
Unsurprisingly, developers have already used these frameworks to successfully implement popular object-oriented programming languages (such as Python, Ruby, or JavaScript), with performance results as good as state of the art \JITing/ \VMs/ \citep{Marr2015,Marr:2016:AWFY}.

Despite the multitude of tailored \VM/s that have been developed using \VM/ frameworks, it appears that developers tend to recreate similar components (a parser, an \AST/ walker, and object storage) \emph{on-top-of} the framework rather than reuse existing implementations from other tailored \VM/s; perhaps doing so in order to gain fine-grained control over later optimizations (at the cost of a significant amount of work).

Recently, the notion of developing a set of components that can be easily extended to create a tailored \VM/ have gained traction. For example, \rcitea{WoB2014}\mwh{The citation style breaks this reference} introduced a robust \emph{Object Storage Model}, which has now been integrated into the Truffle framework; and \rcite{Inostroza2015} introduce reusable components for languages with denotational semantics using \emph{Object Algebras} \rcite{Oliveira2012}. From much the same perspective, we propose to explore the potential to create new \VM/s by extending \VM/s built using \VM/ frameworks but tailored other languages. As an initial experiment, we adapt SOMns to realize a \VM/ for Grace.


%
% ---------------------------------------------------------------------------- %



% ---------------------------------------------------------------------------- %
% Notes

% Stefan - People are playing with reusable components built in VM frameworks
%
% eg Product Lines of Interpreters Using Truffle with Object Algebras
% 	 Modular interpreters for the masses: implicit context propagation using object algebras

% Stefan - Object storage is provided by truffle, so, the fact that SOMns has its own is somewhat an artifact of ‘history’, and perhaps somewhat an optimization/step to avoid dependence on closed-source Oracle features. There are other aspects like storage strategies (think of our arrays), which haven’t been included in truffle yet, but are for instance in RPython: 
% -- http://dl.acm.org/citation.cfm?id=2816716

% Richard - The original idea of a \emph{\VM/} was a tool that compiled a program from an abstract into machine executable code, useful in that developers could implement a method to parse their language into the \ir understood by the \VM/. 

    % Stefan - I don’t think so, p-code might be the first example of a virtual machine. There was no translation, it was just to avoid targeting a specific hardware machine, because back then they had many more different architectures than 3-4 people care about today


% Richard - An interpreter is collections of tools for executing a program. Lexical analysis encodes an intermeditary representation with the logic of the program and then that representation is visited to realize execution. From its outset, practitioners have created a multitude of different interpreters for different languages, each with a toolkit based on similar conventions but unique in its implementation (often required to support the semantics of the language being interpreted). 

% 
% ---------------------------------------------------------------------------- %
