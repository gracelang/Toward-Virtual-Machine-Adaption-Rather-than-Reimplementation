%!TEX root = ../paper.tex

% ---------------------------------------------------------------------------- %
% Outline

\iftoggle{outline}{

    \begin{itemize}
        \item Introduce Grace
        \item Introduce Newspeak
        \item Introduce SOMns
    \end{itemize}

}{}


(kjx: you know, perhpas we don't need to say much here)

Newspeak is an object-oriented langauge designed to combine the
benefits of Smalltalk but supporting deeply nested class definitons to
support libraries and family polymorphism as in BETA and gBETA
\cite{newspeak-modules,betabook,fampoly}.  Like Smalltalk, Newspeak is
a pure object-oriented langauge, in that all computation proceeds via
messages sent to objects. Newspeak also adopts Smalltalk's multi-part
message syntax and use of lambda-expressions (``blocks'') to encode
control structures.

Grace is a new object-oriented language ostensibly designed to support
education \cite{graceOnward12}.  Grace's syntax is designed to look
like Java (or the ``curly bracket'' langauges more generally) but
draws widely on the traditions of object-oriented programming.  Grace
is designed to mix static and dynamic types (like Strongtalk or
C$\sharp$); makes heavy use of lambda expresions (like Smalltalk or
Ruby); supports nested class definitions (like BETA); and uses method
aliasing for inheritance (somewhat like Eiffel). 

Newspeak was a significant influence on Grace's design. Grace shares
with Newspeak that static types generally do not affect a program's
execution; that control strucures are handled with lambda expressions;
and (to an extent) that class definitions can be nested.  Probably the
most signficant underlying difference is that in Grace, classes are
defined in terms of objects, and their definitons can be nested freely
(inside other objects, classes, methods, or lambda expressions)
whereas in Newspeak, obejcts are defined in terms of classes, and
classes can only be declared nested inside other classes.
There are number of Newspeak implementations, here we are interested
on SOMns, a high-performnce implementation based on the Truffle
framework and Graal compiler \cite{stefanEventLoop,Truffle,Graal}.

(Stefan to write something on SOMns :-)



% 
% ---------------------------------------------------------------------------- %


\subsection{Related Work}


\paragraph{\vmframeworks{}} 

\begin{itemize}
\item  \cite{Wurthinger2012} Self optimizing AST
\end{itemize}

\paragraph{Reusable Tools} 

\begin{itemize}
\item \cite{WoB2014} Reusable Object storage model
\end{itemize}
