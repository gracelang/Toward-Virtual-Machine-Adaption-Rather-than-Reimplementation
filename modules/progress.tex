%!TEX root = ../paper.tex

% ---------------------------------------------------------------------------- %
% Outline

\iftoggle{outline}{

    \begin{itemize}
        \item Progress so far
        \item Diagram of components
        \item Benchmarks and description
    \end{itemize}

}

% 
% ---------------------------------------------------------------------------- %

Reusing an existing Grace parser, we translate Grace's \AST/ nodes into those used by SOMns to have a basic Grace \VM/, see \autoref{fig:components}.
Thus, we modified SOMns to use our translator. When called, the translator first parses Grace's prelude and then continues to the input program.
Once an \AST/ has been formed, the translator composes the program representation based on SOMns's Truffle nodes. 

\begin{figure}
    \centering
    \makebox[\columnwidth][c]{
        \resizebox{\columnwidth}{!}{
            
\begin{tikzpicture}[
	edge from parent/.style ={}
]

\node[somns] (NewspeakPrelude) {Newspeak Prelude};
\node[somns, xshift=2cm, right of = NewspeakPrelude] (NewspeakParser) {Newspeak Parser};

\node[somns, xshift=2cm, right of = NewspeakParser] (SOMnsAST) {SOMns AST};
\node[somns, xshift=2cm, right of = SOMnsAST] (Truffle) {Truffle};
\node[somns, xshift=2cm, right of = Truffle, below of = Truffle] (Graal) {Graal};

\node[gracen, yshift=-1.75cm, below of = SOMnsAST] (Translator) {Translator};
\node[grace, yshift=-0.75cm, below of = Translator] (GraceAST) {Grace AST};
\node[grace, xshift=-2cm, left of = GraceAST] (GraceParser) {Grace Parser};
\node[grace, xshift=-2cm, left of = GraceParser] (GracePrelude) {Grace Prelude};

\draw[dashed] (-1, -1.75) -- (12, -1.75);
\node at (1, -1.25) {SOMns};
\node at (1, -2.25) {Grace Support};

\draw[->>, dashed] (NewspeakPrelude) -- (NewspeakParser);
\draw[->>, dashed] (NewspeakParser) -- (SOMnsAST);
\draw[->>, thick] (SOMnsAST) -- (Truffle);
\draw[->>, thick] (Truffle) |- (Graal);
\draw[->>, thick] (Graal) |- (Truffle);

\draw[->>, thick] (GracePrelude) -- (GraceParser);
\draw[->>, thick] (GraceParser) -- (GraceAST);
\draw[->>, thick] (GraceAST) -- (Translator);

\draw[->>, thick] (Translator) -- (SOMnsAST);

\end{tikzpicture}
        }
    }
    \caption{The components of SOMns and those used to adapt the implementation to a Grace \VM/. We used an existing prelude and parser for Grace.}
    \label{fig:components}
\end{figure}

For the initial proof of concept, we implemented expression nodes, numerals and string nodes, as well as nodes to maintain lexical scoping for blocks and methods. The translator is little more than an AST visitor that reads the nodes in one form and exports them in other; and in a few cases it composes together multiple nodes to preserve Grace's semantics. 

\begin{figure}
    \centering
    \makebox[\columnwidth][c]{
        \resizebox{\columnwidth}{!}{
            \begin{tikzpicture}
    \begin{axis}[
        ybar, axis on top,
        height=8cm, width=16cm,
        bar width=0.4cm,
        ymajorgrids, tick align=inside,
        major grid style={draw=gray},
        enlarge y limits={value=.1,upper},
        ymin=0, ymax=15,
        axis x line*=bottom,
        axis y line*=right,
        y axis line style={opacity=0},
        tickwidth=0pt,
        enlarge x limits=true,
        legend style={
            at={(0.5,-0.2)},
            anchor=north,
            legend columns=-1,
            /tikz/every even column/.append style={column sep=0.5cm}
        },
        ylabel={Execution Time (seconds)},
        symbolic x coords={
           1,
           Sieve,
           Sum to Million,
           Mandelbrot,
           1},
       xtick=data,
    ]
    \addplot [draw=none, fill=col-lred] coordinates { % Grace
      (Sieve,           12.225117)
      (Sum to Million,  9.31109999)
      (Mandelbrot,      12.2251173) };
    \addplot [draw=none, fill=col-red] coordinates { % Grace on SOMns
      (Sieve,           81.978964)
      (Sum to Million,   1.678865)
      (Mandelbrot,      13.445174) };
    \addplot [draw=none, fill=col-dred] coordinates {  % Grace on SOMns (G)
      (Sieve,            0.977399)
      (Sum to Million,   0.236139)
      (Mandelbrot,       0.859819) };
    \addplot [draw=none, fill=col-dblue] coordinates { % Newspeak on SOMns (G)
      (Sieve,            0.873562)
      (Sum to Million,   0.213156)
      (Mandelbrot,       0.338762) };
    \addplot [draw=none, fill=col-blue] coordinates {  % Newspeak on SOMns
      (Sieve,           12.225117)
      (Sum to Million,   1.905146)   
      (Mandelbrot,       4.296338) };

    \legend{Grace, Grace (S), Grace (S+G), Newspeak (S+G), Newspeak (S)}
    \end{axis}

    \node at (0.75, 6.75) {81.9};
\end{tikzpicture}
        }
    }
    \caption{The execution time reported by SOMns when executing three integer-based benchmarks: calculating the factorial of 100, calculating the sum the natural numbers from zero up to one million, and calculating $500$ iterations of the Mandelbrot fractal. The implementations of these benchmarks in Grace and Newspeak have only negligible differences. The table depicts the average execution time over $1000$ iterations, with and without the Graal optimization layer.}
    \label{fig:benchmark}
\end{figure}

Despite being tailored to Newspeak, SOMns can be easily adapted to Grace through our translator. To date SOMns has been under development for $22$ months: it is mostly compliant with Newspeak, and is as fast as other \JITing/ \VMs/. In contrast Grace has been in development for around five years, without consideration of performance. In only three months, we have adapted a Truffle-based \VM/ to realize a new implementation of Grace without comparable execution speed -- as depicted in \autoref{fig:benchmark}, there is only a marginal loss of performance due to the additional translation step. In conclusion, the implementation of the translator requires a trivial amount of work but retains much of the optimizations carefully tailored for Newspeak.

We have presented an experiment in which we realized an \VM/ for Grace from SOMns, while retaining much of the \AST/ optimizations. While further experiments are required before we can quantify the extent to which the similarity of the languages, the craft of SOMns, and the flexibility of Truffle contribute to the ability to make such an adaption. 


